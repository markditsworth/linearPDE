\documentclass{amsart}

\usepackage{amsmath}
\usepackage{amssymb}

\title{PDE: Assignment 1}
\author{Mark Ditsworth}

\begin{document}
	\maketitle
	
	\section{Problem 1}
	If $B$ is a positive-definite, Hermitian matrix, show that there is a unique matrix $\sqrt{B}$ that is Hermitian positive-definite and $\left( \sqrt{B} \right) ^2 = B$.
	\\\\
	Let $M$ be the matrix of eigenvectors of $B$. Since $B$ is Hermitian and positive definite, $M$ and it's inverse is Hermitian and positive-definite. Then $B' = M^{-1}BM$, and $B'$ is a diagonal matrix of the eigenvalues of $B$.
	\\\\
	$M\sqrt{B'}M^{-1} = \sqrt{B}$, which is Hermitian and positive-definite.
	\\\\
	Even $M$ is not unique (any $\alpha M$ is a matrix of eigenvectors of $B$), $B'$ is unique since $(\alpha M)^{-1}B\alpha M = \frac{1}{\alpha}\alpha M^{-1}BM$. Therefore, $\sqrt{B}$ is unique.
	\\\\
	\section{Problem 2}
	$A$ and $B$ are Hermitian and $B$ is positive definite.
	\subsection{Part 1}
	Show that $B^{-1}A$ is similar to a Hermitian.
	\\\\
	$C$ is similar to $B^{-1}A \Rightarrow C=MB^{-1}AM^{-1}$ for any invertible $M$. From Problem 1, $\sqrt{B}$ is invertible so let $M=\sqrt{B}$.
	\\\\
	$C = \sqrt{B}B^{-1}A\sqrt{B}^{-1}=B^{-1/2}AB^{-1/2}$, which is Hermitian since $A$ and $\sqrt{B}$ are Hermitian. Therefore, $B^{-1}A$ is similar to a Hermitian.
	\subsection{Part 2}
	What does this say about the eigenvalues of $B^{-1}A$?
	\\\\
	The eigenvalues are real.
	\subsection{Part 3}
	Are the eigenvectors orthogonal?
	\\\\
	No. The eigenvectors of $C$ are orthagonal, but $B^{-1}A$ is not Hermitian. It is \textit{similar} to a Hermitian matrix. So it will not necessarily have orthagonal eigenvectors.
	
	\subsection{Part 4}
	Verify the answers above in Julia with $5\times 5$ matrices.
	\\\\
	\textit{see notebook.}
	\\\\
	\subsection{Part 5}
	What is special about $C=M^{T}BM$? Show that the elements of $C$ are a kind of dot product of the eigenvectors with a factor of B in the middle.
	\\\\
	\textit{see notebook.}
	
	\section{Problem 3}
	The solutions of ODE $y'' - 2y' - cy = 0$ take the form $y(t) = C_1 e^{(1+\sqrt{1+c})t} + C_2 e^{(1-\sqrt{1+c})t}$ for some constants $C_1$ and $C_2$ determined by the initial conditions. Suppose that $A$ is a real-symmetric $4 \times 4$ matrix with eigenvalues 3, 8, 15, and 24, which correspond to eigenvectors $\mathbf{x_1}$, $\mathbf{x_2}$, $\mathbf{x_3}$, $\mathbf{x_4}$.
	\subsection{Part 1}
	If $\mathbf{x}(t)$ solves the system of ODEs $\frac{d^2}{dx^2}\mathbf{x} - 2\frac{d}{dx}\mathbf{x} = A\mathbf{x}$, with initial conditions $\mathbf{x}(0) = \mathbf{a}_0$ and $\mathbf{x}'(0) = \mathbf{b}_0$, find the closed-form expression of $\mathbf{x}(t)$ in terms of the eigenvectors and initial conditions.
	\\\\
	Since A is Hermitian and positive-definite, the eigenvectors are orthogonal. Thus, the solution can be expressed as linear combination of the eigenvectors.
	\[\mathbf{x}(t) = \sum_{n=1}^{4} c_n(t) \mathbf{x}_n \]
	
	\[\frac{d^2}{dx^2}\mathbf{x}(t) - 2\frac{d}{dx}\mathbf{x}(t) - A\mathbf{x}(t) = 
	\sum_{n=1}^{4}\left( \ddot{c_n} - 2\dot{c_n}- \lambda_n \right) \mathbf{x}_n = 0\]
	
	Since $\mathbf{x}_n$ are linearly independent, $\ddot{c_n} - 2\dot{c_n}- \lambda_n = 0$ for all $n$.
	
	\[ c_n(t) = \alpha_n e^{(1+\sqrt{1+\lambda_n})t} + 
	            \beta_n e^{(1-\sqrt{1+\lambda_n})t} \]
	
	\[ \mathbf{x}(0) = \sum_{n=1}^{4} (\alpha_n + \beta_n)\mathbf{x}_n = \mathbf{a}_0\]
	
	\[ \alpha_n + \beta_n = \frac{\mathbf{x}_x^* \mathbf{a}_0}{||\mathbf{x}_n||^2}\]
	
	\[ \mathbf{x}'(0) = \sum_{n=1}^{4} \left( \alpha_n(1+\sqrt{1+\lambda_n}) + \beta_n(1-\sqrt{1+\lambda_n}) \right)\mathbf{x}_n = \mathbf{b}_0 \]
	
	\[ \mathbf{x}'(0) = \sum_{n=1}^{4}\left( \alpha_n + \beta_n + \sqrt{1+\lambda_n}(\alpha_n - \beta_n) \right) \mathbf{x}_n = \mathbf{b}_0 \]
	
	\[ \frac{\mathbf{x}_n^* \mathbf{a}_0}{||\mathbf{x}_n||^2} + \sqrt{1+\lambda_n}(\alpha_n - \beta_n) = \frac{\mathbf{x}_n^* \mathbf{b}_0}{||\mathbf{x}_n||^2} \]
	
	\[ \alpha_n - \beta_n = \frac{\mathbf{x}_n^* (\mathbf{b}_0 - \mathbf{a}_0)}{||\mathbf{x}_n||^2 \sqrt{1+\lambda_n}}\]
	
	\[ \mathbf{x}(t) = \sum_{n=1}^{4} \left( \left(\mathbf{x}_n^*\mathbf{a}_0 + \frac{\mathbf{x}_n^*(\mathbf{b}_0 - \mathbf{a}_0)}{\sqrt{1+\lambda_n}} \right)e^{(1+\sqrt{1+\lambda_n})t} +
	\left(\mathbf{x}_n^*\mathbf{a}_0 - \frac{\mathbf{x}_n^*(\mathbf{b}_0 - \mathbf{a}_0)}{\sqrt{1+\lambda_n}} \right)e^{(1-\sqrt{1+\lambda_n})t}
	\right) \frac{\mathbf{x}_n}{2||\mathbf{x}_n||^2} \]
	
	\subsection{Part 2}
	After a long time ($t >> 0$), what is the expected approximation of the solution?
	\\\\
	$\mathbf{x}(t)$ will be dominated by the fastest growing term in the mode with the largest eigenvalue ($\lambda_4 = 24$).
	
	\[ \mathbf{x}(t>>0) \cong \left( \mathbf{x}_4^* \mathbf{a}_0 + \frac{\mathbf{x}_4^* (\mathbf{b}_0 - \mathbf{a}_0)}{\sqrt{1+\lambda_4}}\right) e^{(1+\sqrt{1+\lambda_4})t} \frac{\mathbf{x}_4}{2||\mathbf{x}_4||^2} \]
	
	\[ \mathbf{x}(t>>0) \cong \left( \mathbf{x}_4^* \mathbf{a}_0 + \frac{\mathbf{x}_4^* (\mathbf{b}_0 - \mathbf{a}_0)}{5}\right) e^{6t} \frac{\mathbf{x}_4}{2||\mathbf{x}_4||^2} \]
	\\\\
	\section{Problem 4}
	Consider the 1d Poisson equation $\frac{d^2}{dx^2}u(x) = f(x)$ for the vector space of functions $u(x)$ on $x \in [0,L]$ with the Dirichlet boundary conditions $u(0) = u(L) = 0$.
	
	\subsection{Part 1}
	Suppose the boundary conditions are changed to \textit{periodic} boundary condition $u(0) = u(L)$. What are the eigenfunctions of $\frac{d^2}{dx^2}$ now? Will Poisson's equations have unique solutions? Under what conditions on $f(x)$ would a solution exist?
	\\\\
	Since the boundary is periodic, the eigenfunctions will be $\sin(kx)$, and $\cos(kx)$, where $k=\frac{2\pi n}{L}$ with $n = 1,2,\dots$ for sine and $n=0,1,2,\dots$ for cosine. 0 is ignored for sine because we do not allow the zero function as an eigenfunction, and negative $n$'s since they are linearly dependent on sine and cosine.
	\\\\
	\noindent
	$\sin(\phi + kx) = \cos(\phi)\sin(kx) + \sin(\phi)\cos(kx)$, and is therefore linearly dependent on sine and cosine. And exponential functions are only periodic for $e^{ikx} = \cos(kx)+i\sin(kx)$, which is linearly dependent on sine and cosine.
	\\\\
	\noindent
	The equation will not have unique solutions since the vector space is spanned by $u(x) = \alpha1 \forall \alpha$ Thus, it can have infinite solutions.
	\\\\
	\noindent
	To solve $\frac{d^2}{dx^2}u(x) = f(x)$, we would divide each term in the Fourier series by it's eigenvalue $(\frac{2\pi n}{L})^2$, which is only defined for $n>0$. This implies the $c_0 = 0$, or equivalently $\int_{0}^{L} f(x) = 0$. Under this condition, the equation is solvable.
	\subsection{Part 2}
	If instead we considered $\frac{d^2}{dx^2}v(x) = g(x)$ with the boundary condition $v(0) = v(L)+1$, do these functions form a vector space?
	\\\\
	A vector space must include the zero function, but if $v(x)=0$, then $v(0) \neq v(L)+1$. Thus, the functions do not form a vector space.
	\subsection{Part 3}
	How can we transform $v(x)$ from Part 2 back into the original $\frac{d^2}{dx^2}u(x) = f(x)$ problem with $u(x) = u(L)$ by writing $u(x)=v(x)+q(x)$ and $f(x)=g(x)+r(x)$ for some function $q$ and $r$?
	\\\\
	\noindent
	Let $q$ be a twice differential function with $q(L)-q(1)=-1$. Then $u(L) - u(0) = (v(L)-v(0))+(q(L)-q(0)) = -1 + 1 = 0$ and $u$ is periodic. Then plug $v = u-q$ back into $\frac{d^2}{dx^2}v(x) = g(x)$:
	
	\[\frac{d^2}{dx^2}(u-q) = g(x)\]
	
	\[ \frac{d^2}{dx^2}u(x) = g(x) + \frac{d^2}{dx^2}q \]
	
	Take $\frac{d^2}{dx^2}q$ as $r$ and
	
	\[ \frac{d^2}{dx^2}u(x) = g(x) + r(x) = f(x)\]
	\\\\
	\section{Problem 5}
	Consider a finite difference approximation as shown below.
	
	\[ u'(x) \approx \frac{-u(x+2\Delta x) + c\cdot u(x+\Delta x) - c\cdot u(x - \Delta x) + u(x - 2\Delta x)}{d\Delta x} \]
	
	\subsection{Part 1}
	Substitute the Taylor Series for $u(x+\Delta x)$ to show that an appropriate choice of $c$ and $d$ will make this approximation 4$^{th}$ order accurate (errors are proportional to $(\Delta x)^2$).
	\\\\
	\[ u(x + \Delta x) = \sum_{n=0}^{\infty}\frac{u^{(n)}(x)(\Delta x)^n}{n!}\]
	
	Denote $\hat{u}'(x)$ as the finite difference approximation of $u'(x)$.
	
	\[ \hat{u}'(x) = \frac{ -\sum_{n=0}^{\infty}\frac{u^{(n)}(x)(2\Delta x)^n}{n!}+
	c\sum_{n=0}^{\infty}\frac{u^{(n)}(x)(\Delta x)^n}{n!}-
	c\sum_{n=0}^{\infty}\frac{u^{(n)}(x)(-\Delta x)^n}{n!}+
	-\sum_{n=0}^{\infty}\frac{u^{(n)}(x)(-2\Delta x)^n}{n!}  }{d\Delta x}\]
	
	\[ \hat{u}'(x) = \frac{
	-2\sum_{n=0}^{\infty}\frac{u^{(2n+1)}(x)(2\Delta x)^{(2n+1)}}{(2n+1)!}+
	2c\sum_{n=0}^{\infty}\frac{u^{(2n+1)}(x)(\Delta x)^{(2n+1)}}{(2n+1)!}
    }{d\Delta x}\]
    
    \[ \hat{u}'(x) = \frac{
    	2\sum_{n=0}^{\infty}\frac{u^{(2n+1)}(x)(\Delta x)^{(2n+1)}}{(2n+1)!}\left( c - 2^{(2n+1)} \right)
    }{d\Delta x}\]

    \[ \hat{u}'(x) = \frac{2}{d} \sum_{n=0}^{\infty} \frac{u^{(2n+1)}(x)(\Delta x)^{2n}}{(2n+1)!}\left( c - 2^{2n+1} \right) 
    \]
    
    \[ \frac{d}{2}\hat{u}'(x) = u'(x)(c-2) + \frac{1}{3!}u^{(3)}(x)(\Delta x)^2(c-8) + \frac{1}{5!}u^{(5)}(x)(\Delta x)^4(c-32) + \dots
    \]
    
    \[ \frac{d}{2}\hat{u}'(x) - u'(x)(c-2) = \frac{1}{3!}u^{(3)}(x)(\Delta x)^2(c-8) + \frac{1}{5!}u^{(5)}(x)(\Delta x)^4(c-32) + \dots
    \]
    \\\\
    Let $c=8$ and $d=12$.
    
    \[ 6\hat{u}'(x) - 6u'(x) = \frac{1}{5!}u^{(5)}(x)(\Delta x)^4(-24) + \dots
    \]
    
    \[ \hat{u}'(x) - u'(x) = \frac{-u^{(5)}(x)}{30} (\Delta x)^4\]
    
    \[\therefore \hat{u}'(x) - u'(x) \propto (\Delta x)^4 \]
    
    \subsection{Part 2}
    Check Part 1 by numerically computing $u'(1)$ for $u(x) = \sin(x)$, as a function of $\Delta x$. Verify the approximation is 4$^th$ order accurate using a loglog plot.
    \\\\
    \textit{see notebook.}
	
\end{document}